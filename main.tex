\documentclass{article}
\usepackage[utf8]{inputenc}
\usepackage{fancyhdr, graphicx, parskip}
\usepackage[spanish]{babel}
\usepackage{listings}
\usepackage{graphicx}
\graphicspath{ {images} }
\usepackage{cite}
\usepackage [colorlinks = true, 
            linkcolor = blue,
            citecolor = black,
            urlcolor = blue]{hyperref}
            

\begin{document}
\begin{titlepage}
\renewcommand{\headrulewidth}{3pt}
\fancyhead[L]{}
\fancyhead[R]{}
    \includegraphics[width=4cm]{imagen.png}
    \begin{center}
        \vspace*{1cm} 
            
        \Huge
        \textbf{Notacion de la memoria del computador}
            
        \vspace{0.8 cm
}
        
        \LARGE
        Taller de la memoria 
            
        \vspace{1.5cm}
            
        \textbf{Jackh Emmanuel Narvaez Guerra}
            
        \vfill
            
        \vspace{0.8cm}
            
        \Large
        Despartamento de Ingeniería Electrónica y Telecomunicaciones\\
        Universidad de Antioquia\\
        Medellín\\
        Septiembre de 2020
            
    \end{center}
\end{titlepage}
\thispagestyle{fancy}

\newpage
\tableofcontents

\include{contenido/cuerpo}
\newpage

\section{Introducción}


se denomina elemento de memoria a cualquier dispositivo que tenga la capacidad de "recordar" información almacenada en el mismo durante un intervalo de tiempo relativamente grande. El elemento básico de información es aquel cuya capacidad es la de recordar un bit información, es decir almacenar un cero o un uno, hasta tanto esa información sea modificada desde el exterior.

Volátiles: se denominan así a aquellas en que su contenido permanece mientras exista tensión que las alimente, la desaparición de la tensión produce la pérdida completa y definitiva de la información almacenada. A este grupo pertenecen las memorias denominadas RAM.

No volátiles: son aquellas en que no se pierde la información incluso cuando desaparezca la tensión de alimentación lo cual significa que si volvemos a alimentar el sistema la información será la misma antes de que se desconectará la tensión y se le denomina ROM.


\section{Defina que es la memoria del computador.}

La memoria cumple un papel muy importante en el computador y su funcionamiento, ya que se trata del dispositivo donde se almacena temporalmente toda la información con la que trabajan
los microprocesadores para procesarla y devolver los resultados que los usuarios requieren.
la memoria de un computador es utilizada para trabajar la información de manera segura y más rápida en donde el usuario puede trabajar sus documentos de manera fácil y hacerle modificaciones sin alterar el documento original, en la memoria se guardan datos que se pueden alterar de manera independiente, pero si no se guarda la información antes de cerrar el programa, la información desaparecerá ya que la memoria solo guarda archivos de manera temporal para tener un mejor rendimiento.


\section{Mencione los tipos de memoria que conozco.} \label{contenido}

Memoria RAM: La memoria RAM es uno de los tipos de memoria mas
importantes en un computador. Ella guarda información en celdas en forma de unos y ceros, la información guardada en dicha memoria se puede acceder de forma rapida y sencilla ya que no guarda la información de forma serial si no
aleatoria.

Disco duro: Un disco duro es aquel que sirve para guardar información de forma casi permanente, este tipo de memoria es necesario para que la información no se pierda y se mantenga de manera segura. Todos los computadores
tienen una integrada en su placa madre ya que sin ella la información quedaría desechable cada vez que el computador se apagara.


\section{Describa la manera como se gestiona la memoria en un computador.}

La gestión de la memoria se hace a través de un microcontrolador que interviene en cada operación y mide la velocidad con la que se transporta la información la cual mide en MHz. El controlados se puede encontrar en una de estas dos partes:
\begin{itemize}
\item{ Un chip situado en la tarjeta madre entre los módulos de memoria y la CPU}
\end{itemize}

\begin{itemize}
\item{ Dentro del microprocesador Este microcontrolador se puede denominar como MCH (Memory Controller Hub o en español, Centro de Control de Memoria).}
\end{itemize}


\section{¿Qué hace que una memoria sea más rápida que otra?}

El disco duro es clave en la velocidad del Disco Duro Cuando se piensa en la capacidad de almacenamiento de un computador, lo primero que viene a la mente es el disco duro. Cuando se piensa en la velocidad del computador, en cambio, lo primero en que se piensa es en el procesador o en la memoria RAM. Es por esto que en el momento de hacer que un computador sea más rápido, lo primero que se suele hacer es ‘aumentar la RAM’. Sin embargo, el principal factor que influye en la velocidad de un computador es, paradójicamente, el disco duro. La razón está en la mecánica de funcionamiento.


\subsection{¿Por qué esto es importante?}

La velocidad determina la rapidez a la que puede trabajar una memoria y afecta junto a su bus de datos, su ancho de banda una mayor velocidad es necesaria dado que para realizar tareas importantes uno necesita un rendimiento mejor y que se haga con una buena velocidad ya que las operaciones de almacenar, borrar y re almacenar nueva información y datos se hará mucho más rápidamente lo que puede marcar la diferencia en su rendimiento



\section{Conclusión} \label{conclulsion}

Las memorias se definen por su similaridad con almacenes internos en el ordenador. El término memoria identifica el almacenaje de datos que viene en forma chips, y el almacenaje de la palabra se utiliza para la memoria que existe en las cintas o los discos. Por otra parte, el término memoria se utiliza generalmente como taquigrafía para la memoria física, que refiere a los chips reales capaces de llevar a cabo datos. Algunos ordenadores también utilizan la memoria virtual, que amplía memoria física sobre un disco duro
\begin{lstlisting}

\end{lstlisting}

A continuación se presenta el logo de C++ Figura (\ref{fig:cpplogo})

\begin{figure}[h]
\includegraphics[width=4cm]{cpplogo.png}
\centering
\caption{Logo de C++}
\label{fig:cpplogo}
\end{figure}

En la sección de teoremas (\ref{contenido})



\bibliographystyle{IEEEtran} 
\bibliography{references}

\end{document}
