\documentclass{article}
\usepackage[utf8]{inputenc}
\usepackage{fancyhdr, graphicx, parskip}
\usepackage[spanish]{babel}
\usepackage{listings}
\usepackage{graphicx}
\graphicspath{ {images} }
\usepackage{cite}
\usepackage [colorlinks = true, 
            linkcolor = blue,
            citecolor = black,
            urlcolor = blue]{hyperref}
            

\begin{document}
\begin{titlepage}
\renewcommand{\headrulewidth}{3pt}
\fancyhead[L]{}
\fancyhead[R]{}
    \includegraphics[width=4cm]{imagen.png}
    \begin{center}
        \vspace*{1cm} 
            
        \Huge
        \textbf{Notacion de la memoria del computador}
            
        \vspace{0.8 cm
}
        
        \LARGE
        Taller de la memoria 
            
        \vspace{1.5cm}
            
        \textbf{Jackh Emmanuel Narvaez Guerra}
            
        \vfill
            
        \vspace{0.8cm}
            
        \Large
        Despartamento de Ingeniería Electrónica y Telecomunicaciones\\
        Universidad de Antioquia\\
        Medellín\\
        Septiembre de 2020
            
    \end{center}
\end{titlepage}
\thispagestyle{fancy}

\tableofcontents
 
\section{Introducción}

se denomina elemento de memoria a cualquier dispositivo que tenga la capacidad de "recordar" información almacenada en el mismo durante un intervalo de tiempo relativamente grande. El elemento básico de información es aquel cuya capacidad es la de recordar un bit información, es decir almacenar un cero o un uno, hasta tanto esa información sea modificada desde el exterior.

Volátiles: se denominan así a aquellas en que su contenido permanece mientras exista tensión que las alimente, la desaparición de la tensión produce la pérdida completa y definitiva de la información almacenada. A este grupo pertenecen las memorias denominadas RAM.

No volátiles: son aquellas en que no se pierde la información incluso cuando desaparezca la tensión de alimentación lo cual significa que si volvemos a alimentar el sistema la información será la misma antes de que se desconectará la tensión y se le denomina ROM.



\section{Defina que es la memoria del computador.}

La memoria cumple un papel muy importante en el computador y su funcionamiento, ya que se trata del dispositivo donde se almacena temporalmente toda la información con la que trabajan
los microprocesadores para procesarla y devolver los resultados que los usuarios requieren.
la memoria de un computador es utilizada para trabajar la información de manera segura y más rápida en donde el usuario puede trabajar sus documentos de manera fácil y hacerle modificaciones sin alterar el documento original, en la memoria se guardan datos que se pueden alterar de manera independiente, pero si no se guarda la información antes de cerrar el programa, la información desaparecerá ya que la memoria solo guarda archivos de manera temporal para tener un mejor rendimiento.


\section{Mencione los tipos de memoria que conozco.} \label{contenido}



\section{Describa la manera como se gestiona la memoria en un computador.}



\section{¿Qué hace que una memoria sea más rápida que otra?}

\subsection{¿Por qué esto es importante?}



\section{Conclusión} \label{conclulsion}


\begin{lstlisting}

\end{lstlisting}

A continuación se presenta el logo de C++ Figura (\ref{fig:cpplogo})

\begin{figure}[h]
\includegraphics[width=4cm]{cpplogo.png}
\centering
\caption{Logo de C++}
\label{fig:cpplogo}
\end{figure}

En la sección de teoremas (\ref{contenido})



\bibliographystyle{IEEEtran} 
\bibliography{references}

\end{document}
