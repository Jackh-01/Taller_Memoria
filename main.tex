\documentclass{article}
\usepackage[utf8]{inputenc}
\usepackage{fancyhdr, graphicx, parskip}
\usepackage[spanish]{babel}
\usepackage{listings}
\usepackage{graphicx}
\graphicspath{ {images} }
\usepackage{cite}
\usepackage [colorlinks = true, 
            linkcolor = blue,
            citecolor = black,
            urlcolor = blue]{hyperref}
            

\begin{document}
\begin{titlepage}
\renewcommand{\headrulewidth}{3pt}
\fancyhead[L]{}
\fancyhead[R]{}
    \includegraphics[width=4cm]{imagen.png}
    \begin{center}
        \vspace*{1cm} 
            
        \Huge
        \textbf{Notacion de la memoria del computador}
            
        \vspace{0.8 cm
}
        
        \LARGE
        Taller de la memoria 
            
        \vspace{1.5cm}
            
        \textbf{Jackh Emmanuel Narvaez Guerra}
            
        \vfill
            
        \vspace{0.8cm}
            
        \Large
        Despartamento de Ingeniería Electrónica y Telecomunicaciones\\
        Universidad de Antioquia\\
        Medellín\\
        Septiembre de 2020
            
    \end{center}
\end{titlepage}
\thispagestyle{fancy}

\newpage
\tableofcontents

\include{contenido/cuerpo}
\newpage

\section{Introducción}


se denomina elemento de memoria a cualquier dispositivo que tenga la capacidad de "recordar" información almacenada en el mismo durante un intervalo de tiempo relativamente grande. El elemento básico de información es aquel cuya capacidad es la de recordar un bit información, es decir almacenar un cero o un uno, hasta tanto esa información sea modificada desde el exterior.

Volátiles: se denominan así a aquellas en que su contenido permanece mientras exista tensión que las alimente, la desaparición de la tensión produce la pérdida completa y definitiva de la información almacenada. A este grupo pertenecen las memorias denominadas RAM.

No volátiles: son aquellas en que no se pierde la información incluso cuando desaparezca la tensión de alimentación lo cual significa que si volvemos a alimentar el sistema la información será la misma antes de que se desconectará la tensión y se le denomina ROM. 


\section{Defina que es la memoria del computador.}

La memoria cumple un papel muy importante en el computador y su funcionamiento, ya que se trata del dispositivo donde se almacena temporalmente toda la información con la que trabajan
los microprocesadores para procesarla y devolver los resultados que los usuarios requieren.
la memoria de un computador es utilizada para trabajar la información de manera segura y más rápida en donde el usuario puede trabajar sus documentos de manera fácil y hacerle modificaciones sin alterar el documento original, en la memoria se guardan datos que se pueden alterar de manera independiente, pero si no se guarda la información antes de cerrar el programa, la información desaparecerá ya que la memoria solo guarda archivos de manera temporal para tener un mejor rendimiento.


\section{Mencione los tipos de memoria que conozco.} \label{contenido}

MEMORIA INTERNA: Con la denominación de memoria interna se pretende hacer referencia a aquellas memorias que pueden ser
accedidas por el CPU en forma cuasi directa, sin mediar módulos de transferencia y comunicación, así como también
a velocidades que se asemejen a la velocidad del CPU.


MEMORIA RAM: La memoria RAM es uno de los tipos de memoria mas
importantes en un computador. Ella guarda información en celdas en forma de unos y ceros, la información guardada en dicha memoria se puede acceder de forma rapida y sencilla ya que no guarda la información de forma serial si no
aleatoria.

DISCO DURO: Un disco duro es aquel que sirve para guardar información de forma casi permanente, este tipo de memoria es necesario para que la información no se pierda y se mantenga de manera segura. Todos los computadores tienen una integrada en su placa madre ya que sin ella la información quedaría desechable cada vez que el computador se apagara.

MEMORIA CACHE: Con el aumento de la rapidez de los microprocesadores ocurrió la paradoja de que las memorias principales no eran suficientemente rápidas como para poder ofrecerles los datos que  ́estos necesitaban. Por esta razón, los ordenadores comenzaron a construirse con un tipo especial de memoria caché interna situada entre el microprocesador y la memoria principal y destinada a almacenar datos que se utilizan frecuentemente. Permite agilizar la transmisión de datos entre el microprocesador y la memoria principal. Es de acceso aleatorio (también conocida como acceso directo) y funciona de una manera similar a como lo hace la memoria principal (RAM), aunque es mucho más rápido

Existen tres tipos de memoria caché cuyo funcionamiento es análogico

\begin{itemize}
\item{L1 o interna al Micro (situada dentro del propio procesador y por tanto de acceso aún más rápido y aun mas
cara). La cache de primer nivel contiene muy pocos kilobytes (unos 32 o 64 Kb).}
\end{itemize}

\begin{itemize}
\item{L2 o externa (situada entre el procesador y la RAM). Los tamaños típicos de la memoria cache L2 oscilan en
la actualidad entre 256 kb y 4 Mb.}
\end{itemize}

\begin{itemize}
\item{L3 esta memoria se encuentra situada en algunas placas base}
\end{itemize}

MEMORIA FLASH: Tipo de memoria que puede ser borrada y reprogramada en unidades de memoria llamadas ”bloques”, en lugar de bytes solos. Los tama ̃nos de los bloques por lo general van de 512 bytes hasta 256 KB. Su nombre se debe aque el microchip permite borrar fragmentos de memoria en una sola acci ́on, o ”flash”. Derivados de EEPROM, los  chips flash son menos costosos y proporcionan mayores densidades de bits. Además, el flash se está convirtiendo en una alternativa para los EPROM porque pueden actualizarse fácilmente. Se utiliza en teléfonos celulares, cámaras digitales y otros dispositivos.

MEMORIA DE DISCOS MAGNETICOS: Este tipo de memoria no volátil tiene su principal uso en el almacenamiento persistente. A diferencia de la memoria ROM la cinta magnética permite escribir nuevamente los sectores y puede conservar su estado incluso cuando no tiene alimentación eléctrica. Algunas memorias de tipo magnético son: Discos duros Memoria usb CD’s Actualmente memorias como los discos duros se emplean para proveer almacenamiento permanente a las computadoras. Existen diversas limitantes para las memorias magnéticas como la cantidad de revoluciones por minuto en discos 4 duros tradicionales o la cantidad de veces que se puede sobre escribir los discos de estado sólido. una característica sobresaliente de este tipo de memoria es su alta capacidad de almacenamiento muy por encima del tipo RAM.

MEMORIA ROM: La memoria ROM solo permite la operación de lectura, de forma que las instrucciones grabadas en ella por
los fabricantes pueden ser utilizadas; pero nunca modificadas; es decir solo se permite la salida de información desde
la memoria hacia el exterior y no al revés. Evidentemente, las memorias de este tipo no son volátiles dado que su
contenido es fijo, y no puede reprogramarse, además si se perdiera la información almacenada en ellas quedarían
inutilizadas.

MEMORIA EPROM: La memoria EPROM es un tipo de chip de memoria ROM que retiene los datos cuando la fuente de energía se
apaga, como en el caso de la memoria ROM, pero a su vez tienen la particularidad de ser programable y borrable, por
lo tanto ser modificada en caso de ser necesario. Se programa mediante impulsos eléctricos y su contenido se borra
exponiéndola a la luz ultravioleta, de manera tal que estos rayos atraen los elementos fotosensibles, modificando su
estado. La memoria EPROM es uno de los tipos de memoria ROM, pero existen otros como la PROM (memoria de
solo lectura programable) y EEPROM (memoria de solo lectura eléctricamente programable).\cite{SILO.TIPS}


\section{Describa la manera como se gestiona la memoria en un computador.}

La gestión de la memoria se hace a través de un microcontrolador que interviene en cada operación y mide la velocidad con la que se transporta la información la cual mide en MHz. El controlados se puede encontrar en una de estas dos partes:
\begin{itemize}
\item{ Un chip situado en la tarjeta madre entre los módulos de memoria y la CPU}
\end{itemize}

\begin{itemize}
\item{ Dentro del microprocesador Este microcontrolador se puede denominar como MCH (Memory Controller Hub o en español, Centro de Control de Memoria).}
\end{itemize}


\section{¿Qué hace que una memoria sea más rápida que otra?}

El disco duro es clave en la velocidad del Disco Duro Cuando se piensa en la capacidad de almacenamiento de un computador, lo primero que viene a la mente es el disco duro. Cuando se piensa en la velocidad del computador, en cambio, lo primero en que se piensa es en el procesador o en la memoria RAM. Es por esto que en el momento de hacer que un computador sea más rápido, lo primero que se suele hacer es ‘aumentar la RAM’. Sin embargo, el principal factor que influye en la velocidad de un computador es, paradójicamente, el disco duro. La razón está en la mecánica de funcionamiento.\cite{SILO.TIPS}


\subsection{¿Por qué esto es importante?}

La velocidad determina la rapidez a la que puede trabajar una memoria y afecta junto a su bus de datos, su ancho de banda una mayor velocidad es necesaria dado que para realizar tareas importantes uno necesita un rendimiento mejor y que se haga con una buena velocidad ya que las operaciones de almacenar, borrar y re almacenar nueva información y datos se hará mucho más rápidamente lo que puede marcar la diferencia en su rendimiento



\section{Conclusión} \label{conclulsion}

Las memorias se definen por su similaridad con almacenes internos en el ordenador. El término memoria identifica el almacenaje de datos que viene en forma chips, y el almacenaje de la palabra se utiliza para la memoria que existe en las cintas o los discos. Por otra parte, el término memoria se utiliza generalmente como taquigrafía para la memoria física, que refiere a los chips reales capaces de llevar a cabo datos. Algunos ordenadores también utilizan la memoria virtual, que amplía memoria física sobre un disco duro.\cite{monografia.com}
\begin{lstlisting}
\end{lstlisting}


\bibliographystyle{IEEEtran}
\bibliography{references}

\end{document}
